\documentclass[a4j,titlepage]{jarticle}

\usepackage[dvipdfmx]{graphicx}
\usepackage[dvipdfmx, hidelinks]{hyperref}

\usepackage{url}

\title{高知工科大学 合気道部 \\ 困った時は...}

\author{kuroneko~nyanten}

\date{\today}

\newpage

%% index

\begin{document}
\maketitle
\tableofcontents
\clearpage


\section{はじめに}
これから記すことは困った時の対処法であり, ある問題に対してこの文書通りにする必要はない. あくまでも, 参考程度のものと考えること. マニュアル通りに行うのではなく, 部員全体で考えて適切な対策を講じることが大切である. \par
本大学では, 合気道は部活動として正式に認められており, 大学側からのサポートを優先的に受けることができる. したがって, サークルや同好会とは全く異なるものである. 開学当初から発足し, そして現在に至っているのは, 他ならぬOB, OGの方々の努力の賜物であり, 今後も合気道部を存続していくためには, 部員各々の自己意識や責任感が重要となってくる. 軽はずみな行動で部員全体に迷惑をかけないように心がけること. \par
また, 稽古の進め方や日程等は幹部(主に主将)が自由に変更しても構わない. 幹部交代は約1年ごとであるが, 前幹部に倣って部活を行う必要はない. 型(稽古方法など)にとらわれず常に変化することも大切である. \par
本学の合気道部は, 正岡師範や舛田先生, OBや一般の方々など, 様々な方々と稽古する機会が多い. 失礼のないよう, 礼儀正しく有意義なものにすること.

\subsection{合気道とは}
合気道は, 武道家・植芝盛平が創始した武道である. 「天地の"気"に合する道」から「合気道」と呼ばれる. \par
筆者自身, まだまだ未熟であるため多くは語れないが, 合気道は格闘技ではない. 自身の精神や心身を鍛錬する武道であると考える. 相手との調和や許し, 解放などが重要となってくる. ただ痛めつけるだけでは, 合気道とは言えないだろう. \par
合気道界では有名な, 塩田剛三に以下のような言葉がある. \par

\vspace{0.2in}
\begin{center}
   “歩けばそれ即すなわち武”
\end{center}
\vspace{0.2in}

つまり, 絶対的な「技」は存在し得ないのである. 合気道はリアルタイムに変化するものであり, 自ら技を創出することが重要であるため, 「型」にとらわれない稽古に精進すべきである.

\subsection{礼儀作法}
本学の合気道部では, 技よりも礼儀作法に重点をおいている. 他大学との稽古や実社会等で重要になってくるため, 正しい礼儀作法を身に付けることは重要である. \par
師範や先生が来られた場合, しっかりと礼ができるようにすること.

\newpage

\section{入部}
合気道部に入部するにあたっては, 入部届が必要になるが, 部室(407号室), または, 正面の裏扉に保管されている. 記入欄に必要事項を記入し主将に提出することで, 正式な部員として認められる. \par
なお, 各部員は月一で部費を払うことになるが, 年度ごとに会計が必要に応じて変更しても構わない. 幹部同士や部員で話し合って決めること. 参考までに, 筆者の年度での部費・会計などは後述する.

\subsection{退部}
やむを得ない理由で退部する場合, 主将に直接話を通すことで退部することは可能である. しかし, 入部した以上, 退部しないことが絶対条件であるため, 入部の際にはよく考えてから入部すること. 師範稽古と学生稽古の両方に参加して, 合気道部が自分に合っているかどうか判断することを推奨する.

\section{稽古}
稽古は毎週4日(内1日が師範稽古, 1日が自主稽古)としている\footnote[1]{平成29年度7月現在.}. 基本的に長期休学期間中は稽古は自主稽古となる. 学生稽古では自主的な稽古が求められる. より多くの稽古に参加することは, より早く昇級・昇段審査を受けることへと繋がる. \par
毎週金曜日の師範稽古は可能であれば必ず参加するという気持ちで望むこと. 正岡師範や舛田先生が来てくださるわずかな時間を無駄にしないよう, 集中して稽古に参加すること. \par
高知工科大学合気道部の練習日程は以下の通りである.

\begin{description}
\item[月曜日]\mbox{} \\
  学生稽古. \\
  時間: 17:00〜19:00 \\
\item[水曜日]\mbox{} \\
  学生稽古. \\
  時間: 18:00〜20:00 \\
\item[金曜日]\mbox{} \\
  師範稽古. \\
  時間: 19:30〜 (杖) \\
  ~~~~~~20:00〜21:30 (師範稽古) \\
\item[土曜日]\mbox{} \\
  自主稽古. \\
  時間: 16:00〜19:00
\end{description}

\newpage

\subsection{準備運動}
準備運動は学生稽古の場合, 各自がそれぞれ行う. 師範稽古は20時から準備運動開始となる. \par
なお, 準備運動に関しても, 前幹部と同じように進める必要はないため, 取り入れたいものがあれば率先的に行うことが望ましい.

\subsection{お迎え・お見送り}
師範稽古日に正岡師範, 舛田先生が来られるであろう, 20時過ぎ(受け練習)頃にお迎えに行くこと. この時間帯はあたりが暗いため, 携帯電話・スマートフォン等の照明器具を持って行く. 見送りに関しても同様である.

\section{特別稽古}
高知工科大学 合気道部では, 様々なイベントが開催される. しかしながら, 他の部活と比較すると決して多い方ではないため, 合同稽古など数少ないイベントには積極的に参加すること.

\subsection{合宿}
合宿は春と夏の長期休学の期間に行う. 夏合宿に関しては, 遠方への泊りがけとなるため, 入念な計画が必要となる. 宿泊先やバスの手配, 師範・舛田先生への連絡など, タスクが多くなるため協力して行うこと. 特に, 宿泊先は他の団体が利用することが多いため, おおよそ1ヶ月前からアポを取っておくことが望ましい. \par
春合宿に関しては, 武産合気凌雲館で行う. 館長(舛田先生)と連絡を取り, スケジュールなどを事前に共有すること.

\subsection{稽古始め/稽古納め}
それぞれの学期(長期休学)前後の師範稽古は稽古納め・稽古始めとなる. この際, 正岡師範に日頃のご指導のお礼として, 粗品と謝礼金を準備すること. 謝礼金については後述する. \par
なお, 繰り返すが, 師範稽古は週に1度である. 1ヶ月を4週間として, 1年間の稽古日程で単純計算すると, 以下のようになる.

\begin{center}
  \begin{math}
    師範稽古 = 4 \times (12ヶ月 - 4ヶ月)
    = 32回
  \end{math}
\end{center}

師範稽古を32回, そして, 引退するまでの3年間を考慮しても少なめであることが伺える. 貴重な師範稽古を大切にしてほしい次第である. \par
筆者自身, 師範稽古を通して多くのことを学び, 実生活に活かせている. やむを得ない理由を除いて, 参加できるのであれば, ぜひ参加してほしい.

\subsection{他大学との合同稽古}
年に数回, 高知県内の各国公立大学が主催とする合同稽古が開催される. 高知大学, 高知県立大学, 高知大学医学部, そして, 高知工科大学がそれぞれ主催となる. 開催時期は主催大学から主将や主務宛てに連絡が来る(主に書類)ため, 必ず参加の有無や人数などを早急に返答すること. \par
また, 合同稽古後に打ち上げが執り行われる場合もある. 他大学との貴重な時間を過ごせるため, 有意義なものにしてもらってかまわない. ただし, 近年増えつつあるアルハラなどには注意すること.

\subsubsection{進め方}
まず, 開催する日程を決める. 執り行われる会場は, 主に高知県立武道館である. 武道館の都合もあるため, 必ず連絡をとり, 他団体となるべく重複しないようにすること. \par
開催場所を確保出来次第, 各大学に文書当てで連絡すること. 主に主務がこれを行う. 

\subsection{インターカレッジ}
インターカレッジ(略称, インカレ. 以下, インカレとする.)は全国の大学が参加する, 高校でいうところのインターハイにあたる. 野球で言えば甲子園である. 合気道にもインカレは存在するが, 合気道は試合が存在しない. そのため, 各地域ごとにいくつかまとまって行う形式であると考えられる(演武を主体とした場合は別であると考えるが, 一般的ではない). \par
高知工科大学は, 四国4県が集まって行う四国地区のインカレにのみ参加する. 主催となるのは, 執り行う会場での国立大学が主催となる場合が多い. 会場は一年ごとの交代制で, 2年連続して開催会場が重複することはない. 2017年現在, 高知県が最後に開催会場となった年は2016年である(2015年は愛媛県, 2017年は徳島県). この法則を考慮すると, 高知県が主催となる年はオリンピック開催時期と一致しているため, 次の開催時期は2020年の東京オリンピックの年であると考えられる. \par

\subsubsection{参加するにあたって}
開催時期が近づくと主催大学から諸連絡が通達される. 各大学の指示にしたがって, 主に主将, 副将が行動すること. 開催日の前日に主将会議(説明会)が開かれるため, 必ず参加して当日の概要を確認すること. また, 近年ではインターネットによるチャットやSNSなどのコミュニティが発達しているため, これらを有効活用して他大学との親密性を築くことも重要である. \par
高知県が主催となった場合, 高知大学が主催となる場合が考えられる. しかしながら, 2017年現在, 高知大学, 高知県立大学の合気道部の部員数が芳しくないため, 部員数の多い高知工科大学が率先して協力を行うことが必要不可欠である. \par
高知大学, 高知県立大学, 高知工科大学の部員数の移行に加え, 高知工科大学経済・マネジメント学群の高知県立大学と弊学が進んでいることもあり, 今後の部員数は全く予想がつかない. 高知大学の部員数減少に伴い, 高知工科大学合気道部の重要性が高まってくるため, 部員確保に全力を尽くすこと. ただし, 人には向き不向きが存在するため, むやみな勧誘は避けるべきである.

\subsubsection{県外主催の場合}
主催会場が県外となった場合, 部員数を考慮してバスの手配をする必要がある. 高知工科大学合気道部は正式な部活動であるため, 大学側の用意するバスの乗車が認められる. インカレの時期が近づくと, 学生支援課から登録幹部に諸連絡が通達される仕組みとなっている. この際, 必要書類がPDF形式で添付される(学生支援課からの諸連絡にはこうした書類が添付されることが多い)ため, 高知工科大学のポータルシステムから確認すること. \par
なお, インカレは合気道部に限ったことではないため, 開催時期が同じ部活動と合同でバスを使用することが多い. 各部活動である程度連絡を取っておくことを勧める.

\subsubsection{県内主催の場合}
高知県が主催会場となった場合, 前述した点に注意しながら執り行うこと. なお, インカレには医学部は参加しないため注意が必要である. 主催会場は, 合同稽古と同様で高知県立武道館にて行う. 正岡師範や舛田先生, 各先生方に連絡を怠らないこと. 県外の大学にも連絡を通達し, 主将会議の会場も連絡を入れること. 会議を行う場所は, 主催大学の教室または会議室などであることが多い. \par
4年間合気道を続ければ, 必ず主催となる年がある. 主催となった場合は, 学年問わず積極的に参加, 協力してインカレを進めていくこと.

\section{昇級/昇段審査}
一定期間稽古を修めると昇級/昇段審査を受けることができる. 目安としては, 合気会の稽古日数を参考にするとよいが, これに順ずる必要はない. 高知工科大学での審査に必要な稽古日数はは少し曖昧である. 高知工科大学での審査は正岡師範がいらっしゃる際に行う. 合否判定は, 正岡師範の判断になる. \par
通常は, 5級から審査を行う. 普段から積極的に稽古を行えば, 不自由なく昇級することが可能であると思われる. 男性はは初段以降から袴を, 女性は5級以降からとなる. この理由については諸説ある(調べてみるとよい).

\subsection{審査用紙の書き方}
昇級/昇段審査には, 審査用紙を提出する必要がある. 審査用紙は体裁にしたがって記述することになるが, 特別書かなくても良い部分もある. そのような場合は, 適宜先輩や舛田先生に聞くなどして間違いのないように"ボールペン"で記述すること. 印鑑も準備し, 間違いのないことを確かめて提出する. 


\subsection{審査技}
昇級/昇段審査では, 自分のやりたい技をしてもらっても構わない. ただし, 合気会の公式ページでは昇級/昇段の条件が決まっているため, 基本的にそちらの技を軸とした審査内容になる. 特に1年生の最初の審査(5級)では, 合気道の基本となる技が主軸となっているため, ある意味では最初の審査が最大の難関とも言える. 


\section{部費}
部活動を継続するにあたり, 部費のやりくりは重要なことである. 例えば, 師範へお礼の粗品や謝礼金を渡す際は部費から賄われる. 短刀や木刀などの用具も部費で補ってゆく他にはない. \par
合気道部は大学の公式な部活動であるため, 年に一度だけ5万円の補助金が支払われるが, それだけでは足りないことを意識しておく必要がある. この際に重要となるのが, 部費である. \par
各部員が支払う部費は年度によって変化する. 部員数にも左右されるため, 幹部同士で話し合って部費を自由に決めてもらって構わない. 筆者が幹部であった際の各学年の部費は, 共通して月500円となっていた. これでも少し怪しい時期があったため, 部員数を考慮して適宜変更することを勧める. \par
また, 合気道部の幹部業務に「会計」業務がある. 会計は合気道部の通帳を常備し, 部の経費のやりくりを行う. この時, 重要となることを以下にまとめる.

\subsection{会計}
会計は部員の部費を集め, 通帳に貯金を行う. その口座に貯金された経費で備品や謝礼金をやりくりする. また, 備品や施設利用費に賄った場合, 領収書を必ず頂くこと. この領収書は会計の書類に使ったり, 使用料金をまとめたりするのに使うため, 領収書をもらう, 使った金額をメモすることを勧める. \par
ただし, 会計業務は事が全てうまくいくとは限らないため, 臨機応変に対応すること. 

\subsection{備品}
合気道部の備品は部室に保管, または, 武道場の二階に保管されている. 杖道や短刀は稽古で使用するため, 大切に扱うこと. ただ, 長年使用されて続けているため, 新たに購入することを勧める. 

\subsection{謝礼金}
合宿や稽古納めなどで, 師範に謝礼金を支払うことがある. 普段の稽古のお礼という形であり, 粗品と合わせて稽古後に渡す. 

\section{幹部業務}
主将, 副将は主に部活動の運営を勧める重要な役割である. 主将は他大学との連携, 合同稽古や稽古納めの日程の取りまとめ, 副将は主将のアシストを主に行う. 主務は, 他大学との手紙によるやりとりなどを行う. 会計は上記で述べたことを主に行う. 

\section{Webサイト・SNS運営}
副将で余裕がある人, または特にすることのない(要は暇な空き時間がある人)はWebサイトやSNSの運営, 書類の作成などをするべきであると考える. 大学で学んだことを生かして, 多方面に渡って合気道部を推し進めてほしい次第である. 筆者は情報学群であり, このマニュアル作成も情報学群で学習した\LaTeX を活用している. \par
Webサイトは元々, 筆者の前幹部である主務の方が作成してくださったものである. 半年以上更新のないまま放置されていたため, 筆者が再開し始めた. これからインターネットを活用した情報発信が増加してくるものと思われる. 幹部の中で更新を行う者を決めたりするなど, 今後も継続してほしい所存である. \par
実際, 筆者が更新を頻繁に行なった結果, 第10代目のOBの方からTwitterを通してご連絡をいただいた. 高知工科大学合気道部の動向がどのようなものなのか注視されている方は大いに存在する. 卒業されたOBの方々への活動報告は重要なものであることがわかる.

\subsection{管理方法}
ID・パスワードは他者に漏らさないよう, 幹部が交代する際に次期幹部へと受けわたす方式とする. パスワードは定期的に変更することを勧める. WebサイトのURLは以下の通りである.

\vspace{5mm}
\href{http://kut-aikido.strikingly.com}{http://kut-aikido.strikingly.com}
\vspace{5mm}

このWebサイトにログインすると, Webの編集を行うページが表示される. 何かしらのイベントがあった場合には更新を行うこと. なお, 記事の投稿には羽目を外しすぎないように注意する. 更新後はTwitterで更新の告知を行うこと. Twitter, Googleのアカウントは以下の通りである. \par

\vspace{5mm}
Twitter: \href{http://twitter.com/kut\_aikido}{@kut\_aikido} \par
Google: kut.aikido@gmail.com \par
\vspace{5mm}

なお, 普段の練習風景などをTwitterで投稿するのは自由である. 有効活用して, 合気道のコミュニティを広げることも良いだろう. \par
また, Webサイトにはカレンダーが存在するが, Webサイトから直接変更を加えることはできない. そのため, 一旦Webページを閉じて, Googleへアクセスすることが必要になる. Googleアカウントも同様に幹部から受け継ぐ形で使用できる. \par
Googleからカレンダーを選択して変更を行えば, Webサイトのカレンダーも自動的に変更される. ただし, Webサイトでも日程を表示することができるため, カレンダーの変更よりはそちらを優先すべきである. \par
卒業されたOB, OGの方々が合気道部の動向をチェックしていることから, インターネットを用いて情報発信は重要な役割を果たしていることが伺える. 手の空いている者が積極的に更新してくれることを願う. 

\section{さいごに}
以上, 長々と説明したが, このマニュアルはあくまでも一例である. 師範が普段おっしゃっているように, マニュアルに従う必要はないため, このマニュアルは参考程度にしてほしい.

\appendix
\section{ほか}

ソースコードはGitHubに公開済み. \par
\href{https://github.com/nyanten/KUT\_Aikido}{https://github.com/nyanten/KUT\_Aikido}\par

合気道部GoogleアカウントのGoogleDriveにも保存されている. \par
\TeX の縦書き小論文サンプル, 審査技一覧など.

\end{document}
